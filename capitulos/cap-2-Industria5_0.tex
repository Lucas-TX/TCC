%%%% CAPÍTULO 2 - REVISÃO DA LITERATURA (OU REVISÃO BIBLIOGRÁFICA, ESTADO DA ARTE, ESTADO DO CONHECIMENTO)
%%
%% O autor deve registrar seu conhecimento sobre a literatura básica do assunto, discutindo e comentando a informação já publicada.
%% A revisão deve ser apresentada, preferencialmente, em ordem cronológica e por blocos de assunto, procurando mostrar a evolução do tema.
%% Título e rótulo de capítulo (rótulos não devem conter caracteres especiais, acentuados ou cedilha)

\chapter{Indústria 5.0}\label{cap:indutria5_0}
\section{Histórico das Revoluções Industriais}

As transformações industriais ao longo da história moldaram a estrutura produtiva da sociedade.
A Primeira Revolução Industrial, no final do século XVIII, foi impulsionada pela energia a vapor, marcando a transição da produção manual para o modelo fabril mecanizado.
A Segunda Revolução Industrial, no século XIX, trouxe a eletricidade e a produção em massa, elevando drasticamente a produtividade.
A Terceira Revolução Industrial, a partir da década de 1970, integrou eletrônica, automação e tecnologias da informação e comunicação, levando à produção automatizada.
A Quarta Revolução Industrial, ou Indústria 4.0, focou na interconectividade de sistemas físicos e digitais, utilizando tecnologias como \gls{IoT} e computação em nuvem e inteligência artificial \cite{VALETTE2023}.
Essa fase da revolução industrial buscou automação avançada, decisões autônomas e customização em massa.
Contudo, a Indústria 4.0 foi criticada por sua abordagem predominantemente tecnológica, negligenciando aspectos humanos, sociais e ambientais dos sistemas produtivos.
Essa falha na aboradagem abriu caminho para o surgimento da Indústria 5.0, que busca reintroduzir valores como a centralidade humana, a sustentabilidade e a resiliência \cite{Xu2021, PIZON2023}.

\section{Limitações da Indústria 4.0}

A Indústria 4.0, embora tenha impulsionado a eficiência e a produtividade por meio da digitalização e automação, tem sido criticada por sua abordagem predominantemente tecnológica, que muitas vezes negligencia os aspectos humanos, sociais e ambientais dos sistemas produtivos.
A marginalização do papel do trabalhador é uma das principais limitações, com sistemas de planejamento e controle que priorizam a automação em detrimento de fatores cognitivos e perceptivos humanos \cite{RANNERTSHAUSER2022}.
Além disso, a Indústria 4.0 apresenta lacunas em relação à sustentabilidade e resiliência. Apesar dos ganhos de produtividade, seu modelo tecnológico não considerou a preservação ambiental e a capacidade de adaptação a crises sistêmicas, como evidenciado pela pandemia de COVID-19 \cite{euCommission2021, Khan2023}.

Diante desses desafios, a Indústria 5.0 surge como uma evolução que busca resgatar o protagonismo humano e ampliar o foco para além da eficiência econômica.
O objetivo é criar sistemas produtivos que não apenas automatizem tarefas, mas que também fortaleçam a criatividade, a personalização e a inclusão.
Essa mudança de enfoque está fundamentada em três pilares centrais: centralidade no ser humano, sustentabilidade ambiental e resiliência organizacional \cite{euCommission2021, Nahavandi2019}.

\section{Modelo da Indústria 5.0}

A Indústria 5.0 representa uma evolução do modelo industrial, transcendendo o foco exclusivo na eficiência e produtividade da Indústria 4.0 para incorporar valores como a centralidade no ser humano, a sustentabilidade e a resiliência \cite{euCommission2021, Xu2021}.
Essa nova fase da indústria busca harmonizar o avanço tecnológico com o bem-estar social e a proteção ambiental \cite{Nahavandi2019}.

A centralidade no ser humano implica em um redesenho dos sistemas produtivos onde o trabalhador é visto como um ativo valioso, e não apenas como um custo.
A tecnologia deve servir para ampliar as capacidades humanas, promover ambientes de trabalho seguros e inclusivos, e garantir que as decisões sejam tomadas com consideração ética e respeito à dignidade humana \cite{Nahavandi2019, TOTH2023}.
Isso se manifesta na colaboração humano-máquina, onde robôs e sistemas de IA atuam como assistentes, potencializando a criatividade e o julgamento humano em tarefas complexas \cite{VALETTE2023}.

A sustentabilidade, por sua vez, orienta a Indústria 5.0 para a criação de processos produtivos que respeitem os limites planetários.
Isso envolve a adoção de tecnologias verdes, a promoção da economia circular, a redução do consumo de energia e recursos, e a minimização de resíduos e emissões.
O objetivo é garantir que a produção industrial contribua para um futuro mais verde e equitativo, alinhando o desenvolvimento econômico com a responsabilidade ambiental \cite{euCommission2021, silva2024}.

A resiliência é o terceiro pilar, capacitando as indústrias a resistirem e se adaptarem a choques e interrupções, como crises geopolíticas, pandemias ou desastres naturais.
Isso requer cadeias de suprimentos mais robustas e flexíveis, capacidade de produção adaptável e processos de negócios ágeis.
A Indústria 5.0 busca construir sistemas que possam manter sua funcionalidade e se recuperar rapidamente diante de adversidades, garantindo a continuidade das operações essenciais \cite{euCommission2021, Khan2023}.

Em suma, a Indústria 5.0 não é uma substituição da Indústria 4.0, mas um complemento que adiciona uma dimensão de valores, focando em como a tecnologia pode ser utilizada para alcançar objetivos sociais e ambientais mais amplos, além dos ganhos de eficiência.
Ela representa uma mudança de paradigma de uma abordagem puramente tecnológica para uma abordagem orientada por valores, onde a inovação e a pesquisa são direcionadas para o serviço à humanidade dentro dos limites planetários \cite{Xu2021, VALETTE2023}.

\section{Indicadores e Métricas da Indústria 5.0}

A mensuração do progresso em direção à Indústria 5.0 requer a formulação de indicadores que ultrapassem os tradicionais parâmetros técnico-econômicos da Indústria 4.0.
Isso implica considerar dimensões como centralidade no ser humano, sustentabilidade e resiliência organizacional \cite{euCommission2021, Nahavandi2019}.
Ao contrário da Indústria 4.0, que priorizava a eficiência operacional, a Indústria 5.0 demanda um novo conjunto de métricas capazes de captar aspectos qualitativos e contextuais.
Indicadores como tempo de ciclo das máquinas ou taxa de falhas continuam relevantes, mas são complementados por medidas subjetivas, como bem-estar dos trabalhadores, aceitação da IA e confiança nos sistemas \cite{TOTH2023}.

Autores como \citeonline{TOTH2023} propõem a arquitetura colaborativa I5arc, que organiza os indicadores da Indústria 5.0 em seis domínios de melhoria contínua, com base em processos colaborativos entre humanos, inteligência artificial e sistemas ciberfísicos.


\section{Relação entre Sociedade 5.0 e Indústria 5.0}

A Sociedade 5.0, um conceito originado no Japão, representa uma visão de futuro na qual a transformação digital é orientada para o bem-estar social, integrando o mundo físico e o ciberespaço para resolver desafios sociais por meio da tecnologia.
Enquanto a Sociedade 5.0 abrange amplos setores como educação, mobilidade, saúde e segurança, a Indústria 5.0 aplica princípios semelhantes ao contexto produtivo, focando na reorganização dos sistemas de manufatura com base em valores \cite{Santos2025,Xu2021}.

Ambos os conceitos compartilham a centralidade no ser humano, a sustentabilidade e a valorização da tecnologia como meio para um fim social, e não como um fim em si mesma.
No entanto, operam em escalas e escopos distintos: a Sociedade 5.0 é uma estratégia nacional abrangente, dependente de políticas públicas e ações coordenadas entre governos, universidades e sociedade civil.
Já a Indústria 5.0 é um desdobramento setorial, que exige o redesenho de práticas organizacionais e modelos de gestão específicos para o ambiente fabril \cite{PIZON2023, TOTH2023}.

A Indústria 5.0 pode ser vista como uma interface operacional da Sociedade 5.0 no domínio industrial, buscando personalização, sustentabilidade e resiliência dentro dos sistemas de produção.
Essa convergência dentro dos sistemas de produção indica uma tendência de evolução sistêmica, onde tecnologia, sociedade e indústria caminham juntas para promover não apenas avanços técnicos, mas também valor social agregado, superando as limitações dos modelos anteriores que priorizavam apenas a eficiência econômica \cite{VALETTE2023}.

\section{Pilares da Indústria 5.0}
\subsection{Centralidade no ser humano}

A centralidade no ser humano é um dos pilares fundamentais da Indústria 5.0, diferenciando-a da Indústria 4.0, que focava na automação e digitalização para ganhos de produtividade.
A Indústria 5.0 propõe um redesenho dos sistemas industriais com ênfase na valorização das capacidades humanas, na sustentabilidade e na resiliência dos processos produtivos \cite{VALETTE2023, euCommission2021}.

Essa abordagem focda no ser humano visa resgatar o papel ativo do operador nos ambientes fabris, não apenas como executor, mas como cocriador e agente decisório.
A literatura aponta a necessidade de sistemas colaborativos entre humanos, inteligência artificial e dispositivos ciberfísicos, promovendo uma sinergia entre as capacidades cognitivas humanas e os recursos tecnológicos avançados \cite{TOTH2023, Santana_2023}.
Modelos como o Human-in-the-loop Cyber-Physical Systems e o Operator 4.0 refletem essa tendência, onde o humano permanece no centro das decisões, especialmente em tarefas que exigem criatividade, julgamento moral, sensibilidade contextual e solução de problemas não estruturados \cite{VALETTE2023, RANNERTSHAUSER2022}.

O pilar da centralidade no ser humano também demanda um redesenho das arquiteturas organizacionais e tecnológicas, com a incorporação de dispositivos usáveis, gêmeos digitais, realidade aumentada e tecnologias explicáveis de IA, com interfaces que garantam que o controle permaneça nas mãos dos operadores \cite{TOTH2023, YANG2024}.
A Indústria 5.0 valoriza atributos humanos como criatividade, pensamento crítico, empatia, julgamento ético e adaptabilidade, que são indispensáveis para lidar com problemas não estruturados e interpretar contextos complexos \cite{RANNERTSHAUSER2022, Nahavandi2019}.

O papel do operador vai além da execução de tarefas, esperando-se que ele participe ativamente do processo decisório, proponha melhorias e interaja com sistemas baseados em inteligência artificial de forma ética e explicável \cite{TOTH2023, PIZON2023}.
Além disso, o desenvolvimento de competências emocionais e sociais é essencial para a criação de ambientes industriais inclusivos e saudáveis, promovendo o bem-estar psicológico, a comunicação efetiva e a cooperação entre equipes humanas e agentes tecnológicos \cite{Santana_2023}.
% Ribeiro2024

\subsection{Sustentabilidade}

A sustentabilidade é um dos pilares fundamentais da Indústria 5.0, assumindo uma posição de destaque em relação aos paradigmas industriais anteriores.
Enquanto a Indústria 4.0 concentrou-se na eficiência operacional e na digitalização, muitas vezes sem considerar de forma abrangente os impactos ambientais, a Indústria 5.0 propõe uma abordagem mais equilibrada, na qual o desenvolvimento tecnológico está alinhado à preservação dos recursos naturais e ao respeito aos limites planetários \cite{VALETTE2023, silva2024, Rame2024}.

Segundo a Comissão Europeia, a Indústria 5.0 reconhece a necessidade de sistemas produtivos resilientes e ambientalmente responsáveis, capazes de enfrentar desafios como as mudanças climáticas, a escassez de recursos e a instabilidade nas cadeias globais de suprimentos.
Nesse contexto, o uso de tecnologias verdes torna-se essencial para mitigar os impactos ambientais da produção industrial, promovendo a economia circular e a redução da pegada de carbono \cite{Rame2024}.

As tecnologias verdes envolvem inovações que minimizam o consumo de energia e matérias-primas, reduzem emissões e resíduos e promovem ciclos de produção mais circulares.
Entre os exemplos estão a manufatura aditiva com materiais biodegradáveis, sistemas de recuperação de energia, sensores de eficiência energética, além de processos produtivos orientados por inteligência artificial que otimizam recursos em tempo real \cite{TOTH2023, silva2024}.
A integração da sustentabilidade ao design dos sistemas industriais requer não apenas tecnologias apropriadas, mas também uma reestruturação dos modelos de negócios e métricas de desempenho, vinculando a criação de valor econômico ao impacto social e ecológico das operações industriais \cite{Santos2025}.

\subsection{Resiliência}

A resiliência é um componente estrutural da Indústria 5.0, cuja proposta é capacitar os sistemas industriais a resistirem, adaptarem-se e se recuperarem rapidamente de perturbações internas ou externas.
Em contraste com a Indústria 4.0, que priorizou a eficiência operacional e a hiperconectividade, a Indústria 5.0 enfatiza a capacidade dos sistemas de produção em manter sua funcionalidade diante de eventos adversos, como pandemias, crises geopolíticas, escassez de insumos ou falhas tecnológicas \cite{euCommission2021, VALETTE2023, Khan2023}.

A resiliência industrial envolve a integração de mecanismos técnicos, organizacionais e humanos capazes de detectar vulnerabilidades, responder a falhas e ajustar-se de maneira autônoma ou assistida.
Essa abordagem requer modelos de produção mais flexíveis, redes de suprimentos diversificadas e estruturas organizacionais descentralizadas, com maior autonomia local para tomada de decisão \cite{Santos2025, silva2024}.
Segundo \citeonline{TOTH2023}, a arquitetura colaborativa proposta pela Indústria 5.0 amplia a resiliência ao promover a participação ativa dos trabalhadores em processos de inovação e adaptação contínua, com o apoio de tecnologias inteligentes que fornecem suporte decisório em tempo real.
A presença de humanos no ciclo de controle favorece diagnósticos contextuais mais precisos e respostas mais eficazes frente à incerteza \cite{TOTH2023}.

Além disso, o uso de tecnologias como gêmeos digitais, realidade aumentada, inteligência artificial explicável e sistemas distribuídos permite antecipar riscos, simular cenários e reagir de forma ágil a mudanças imprevistas.
Esses elementos tornam-se cruciais para enfrentar ambientes voláteis, incertos, complexos e ambíguos, especialmente em cadeias globais de valor \cite{VALETTE2023}.
A resiliência na Indústria 5.0 não deve ser compreendida apenas como a capacidade tecnológica de resistir a choques externos, mas como uma propriedade sistêmica construída a partir de arranjos organizacionais, culturais e humanos \cite{TOTH2023, Santos2025}.

Um dos elementos centrais para a resiliência organizacional é a promoção de uma cultura de aprendizado contínuo.
Ambientes industriais que valorizam a atualização constante de habilidades, a experimentação controlada e a retroalimentação entre operadores e sistemas inteligentes tornam-se mais preparados para responder a mudanças abruptas \cite{TOTH2023}.
Outro fator chave é a existência de redes de cooperação dentro e fora da organização, fortalecendo a troca de informações, a antecipação de riscos e a inovação distribuída \cite{VALETTE2023, Santana_2023}.

Além disso, a descentralização decisória é apontada como um componente essencial da resiliência organizacional, permitindo que organizações ajustem seus processos de forma mais ágil e contextualizada \cite{PIZON2023, Nahavandi2019}.
Essa descentralização decisória está alinhada ao conceito de \textit{human-in-the-loop}, em que o operador não apenas executa, mas interpreta, adapta e contribui para o sistema produtivo. A integração de tecnologias como gêmeos digitais, \gls{IoT} e \gls{IAx} deve, portanto, ser orientada por arquiteturas organizacionais que respeitem e potencializem a agência humana.

Assim, a resiliência na Indústria 5.0 depende tanto da incorporação de tecnologias quanto da criação de culturas organizacionais que favoreçam a aprendizagem contínua, a colaboração e a autonomia.
Dessa forma, a resiliência um atributo sociotécnico fundamental para a sustentabilidade e adaptabilidade dos sistemas industriais contemporâneos.

\section{Tecnologias aderentes a Indústria 5.0}

Como a colaboração entre seres humanos e máquinas é um dos eixos centrais da Indústria 5.0,  a Indústria 5.0 propõe um modelo de cooperação sinérgica, onde humanos e sistemas inteligentes exploram suas capacidades complementares \cite{Nahavandi2019, Santana_2023}.
Essa colaboração humano-máquina envolve operadores e tecnologias como robôs colaborativos, \gls{IA}, sistemas ciberfísicos e dispositivos usáveis, exigindo arquiteturas cognitivas e organizacionais que garantam eficiência, segurança, adaptabilidade e inteligibilidade \cite{TOTH2023, PIZON2023}.

Estudos indicam que a cognição humana é indispensável em atividades que demandam julgamento situacional, interpretação subjetiva e criatividade, enquanto as máquinas oferecem precisão e velocidade.
A chave da colaboração reside na orquestração dessas competências humanas em processos de coexecução e coconstrução de conhecimento \cite{TOTH2023} com máquinas.
A arquitetura colaborativa proposta por \citeonline{TOTH2023} enfatiza sistemas baseados em ontologias e aprendizado contextualizado, permitindo que operadores interajam com interfaces amigáveis, tomem decisões assistidas por \gls{IA} e personalizem sua experiência de trabalho, tornando-se cocriadores no processo produtivo \cite{TOTH2023, YANG2024}.

A colaboração homem-máquina é uma resposta estratégica à crescente complexidade dos sistemas produtivos, e sua adoção bem-sucedida requer o redesenho dos ambientes industriais para favorecer a inclusão, a aprendizagem contínua e a participação ativa dos trabalhadores nos processos decisórios e inovativos \cite{silva2024}.

\subsection{Inteligência Artificial Centrada no Humano}

A \gls{IA} centrada no humano é um dos principais alicerces tecnológicos da Indústria 5.0.
Diferentemente dos modelos anteriores que priorizavam a automação e substituição da mão de obra humana, a \gls{IA} na Indústria 5.0 é concebida para atuar de forma complementar aos operadores, promovendo colaboração, explicabilidade dos algoritimos e controle humano no ciclo de decisão \cite{TOTH2023, PIZON2023}.
O modelo da Indústria 5.0 se apoia na ideia de que sistemas inteligentes devem ser projetados para ampliar as capacidades cognitivas e perceptivas dos trabalhadores, respeitando seus limites e valores. Isso inclui o desenvolvimento de interfaces intuitivas, algoritmos explicáveis como a \gls{IAx}  e mecanismos de supervisão contínua por parte dos usuários humanos \cite{TOTH2023, VALETTE2023}.

A arquitetura colaborativa I5arc, proposta por \cite{TOTH2023}, exemplifica a concepção de centralidade no ser humano, ao integrar ferramentas de cocriação e coexecução entre humanos e \gls{IA}, com base em conhecimento semântico e linguagens acessíveis para todos os agentes envolvidos.
O modelo propõe o uso de tecnologias como óculos inteligentes, ontologias industriais e gêmeos digitais, permitindo uma colaboração simbiótica entre operadores e sistemas autônomos.
O conceito de \gls{IA} centrada no humano também é estratégico para garantir a aceitação das tecnologias em ambientes industriais, reduzindo resistências e promovendo confiança.
Estudos mostram que trabalhadores demonstram maior adesão quando percebem que têm controle sobre as decisões assistidas por \gls{IA}, especialmente em tarefas críticas de produção, manutenção e qualidade \cite{Santana_2023,Sousa2024}.

\subsection{Internet das Coisas e Sistemas Ciberfísicos}

A \gls{IoT} e os Sistemas Ciberfísicos constituem a base técnica para a integração entre o mundo físico e o digital nas arquiteturas produtivas da Indústria 5.0.
Essas tecnologias já desempenharam papel fundamental na Indústria 4.0, viabilizando a conectividade em tempo real entre máquinas, sensores, dispositivos e sistemas corporativos.
Na Indústria 5.0, no entanto, seu uso é reorientado para permitir interações mais inteligentes, colaborativas e adaptativas entre humanos e tecnologia \cite{VALETTE2023, PIZON2023}.

A IoT permite o monitoramento contínuo de ativos físicos por meio de sensores distribuídos, possibilitando a coleta e análise de grandes volumes de dados, também chamado de \textit{big data} com elevada granularidade.
Já os sistemas ciberfísicos integram esses dados ao ambiente digital por meio de modelos computacionais que representam e controlam os processos físicos em tempo real \cite{TOTH2023}.
Na Indústria 5.0, o papel desses sistemas vai além da automação: eles são utilizados para criar ambientes responsivos que se adaptam às decisões humanas e às mudanças contextuais.
Por exemplo, sensores podem ajustar condições de operação conforme a ergonomia do operador, ou um sistema ciberfísico pode priorizar tarefas com base em parâmetros éticos ou socioambientais definidos por humanos \cite{TOTH2023}.
A combinação de \gls{IoT} e sistemas ciberfísicos habilita aplicações como gêmeos digitais, manutenção preditiva, rastreabilidade, produção sob demanda e personalização em massa.
Além disso, tornam-se essenciais para garantir resiliência operacional e segurança, especialmente em ambientes industriais dinâmicos e sensíveis ao tempo \cite{VALETTE2023}.

\subsection{Robôs Colaborativos}

A introdução de robôs colaborativos representa uma das principais inovações da Indústria 5.0 no que se refere à integração entre humanos e sistemas automatizados.
Diferentemente dos robôs industriais tradicionais, projetados para operar de forma isolada, os robôs colaborativos são desenvolvidos para atuar lado a lado com operadores humanos, compartilhando o mesmo espaço de trabalho e interagindo fisicamente de maneira segura e eficiente \cite{PIZON2023, TOTH2023}.
A interação homem-máquina nesse novo paradigma exige a adoção de princípios ergonômicos, cognitivos e sociais no design de sistemas produtivos.
Isso implica a criação de interfaces intuitivas, sensoriamento avançado, controle por voz e gestos, bem como protocolos de segurança que permitam a detecção e prevenção de colisões, fadiga ou erros operacionais \cite{TOTH2023, YANG2024}.

Segundo \citeonline{TOTH2023}, a arquitetura de colaboração da Indústria 5.0 propõe a orquestração de múltiplos agentes, como humanos, robôs, \gls{IA} e dispositivos \gls{IoT}, em processos de cocriação e coexecução.
Nessa configuração de arquitetura, os cobots não apenas executam tarefas repetitivas, mas também aprendem e se adaptam com base no comportamento e nas decisões humanas. \citeonline{Santana_2023} apontam que a confiança na tecnologia, a explicabilidade das decisões dos sistemas automatizados e a clareza dos papéis desempenhados por humanos e máquinas são fatores críticos para o sucesso dessa colaboração.
A perspectiva da Indústria 5.0, portanto, vai além da eficiência: busca promover ambientes produtivos nos quais a tecnologia seja um amplificador das capacidades humanas, e não um substituto.


