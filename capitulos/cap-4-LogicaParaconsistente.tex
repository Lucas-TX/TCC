%%%% CAPÍTULO 2 - REVISÃO DA LITERATURA (OU REVISÃO BIBLIOGRÁFICA, ESTADO DA ARTE, ESTADO DO CONHECIMENTO)
%%
%% O autor deve registrar seu conhecimento sobre a literatura básica do assunto, discutindo e comentando a informação já publicada.
%% A revisão deve ser apresentada, preferencialmente, em ordem cronológica e por blocos de assunto, procurando mostrar a evolução do tema.
%% Título e rótulo de capítulo (rótulos não devem conter caracteres especiais, acentuados ou cedilha)

\chapter{Lógica Paraconsistente}
\label{cap:logica-paraconsistente}

\section{Origens e fundamentos filosóficos}
\subsection{O problema da contradição na lógica clássica}
\subsection{Contribuições iniciais de Newton da Costa}
\subsection{O papel da Lógica Paraconsistente no pensamento não clássico}

\section{Conceitos fundamentais da Lógica Paraconsistente}
\subsection{Definições gerais}
\subsection{Sistemas lógicos paraconsistentes}
\subsection{Comparativo com outras abordagens não clássicas (resumido)}

\section{A Lógica Paraconsistente Anotada de Dois Valores (LPA2v)}
\subsection{Definição e estrutura da LPA2v}
\subsection{Graus de evidência favorável e desfavorável}
\subsection{Plano cartesiano da LPA2v}
\subsection{Regiões de decisão e interpretação dos resultados}
\subsection{Aplicações da LPA2v em problemas de decisão}

\section{Lógica Paraconsistente e Indústria 5.0}
\subsection{Justificativa do uso da LPA2v para avaliação de maturidade}
\subsection{Vantagens frente a métodos tradicionais multicritério}
\subsection{Relação entre incerteza, contradição e sistemas inteligentes}

\section{Considerações finais}
\subsection{Síntese dos principais conceitos apresentados}
\subsection{Relevância da LPA2v para aplicações industriais e tecnológicas}
