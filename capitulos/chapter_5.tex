\chapter{Avaliação de Maturidade na Indústria 5.0}


Modelos de maturidade organizacional (MMs) são ferramentas estratégicas que permitem às organizações avaliar seu estado atual em relação a um conjunto de capacidades ou processos, fornecendo um roteiro para o aprimoramento contínuo [Angreani2020]. Esses modelos são amplamente utilizados em diversas áreas, incluindo gestão de projetos, desenvolvimento de software e, mais recentemente, na avaliação da prontidão para a Indústria 4.0 e 5.0 [Schumacher2016].

Um MM tipicamente descreve uma sequência de níveis, que representam estágios de evolução de uma capacidade específica. Cada nível é caracterizado por um conjunto de atributos, práticas e resultados que uma organização deve demonstrar para ser classificada naquele estágio. A progressão através dos níveis indica um aumento na sofisticação, eficiência e previsibilidade dos processos [Mittal2018].

Os objetivos primários de um modelo de maturidade incluem [Angreani2020]:
\begin{itemize}
    \item Fornecer uma estrutura para auditoria e benchmarking, permitindo que as organizações comparem seu desempenho com as melhores práticas ou com outras entidades.
    \item Medir o progresso ao longo do tempo, identificando o quão longe a organização avançou em direção aos seus objetivos de melhoria.
    \item Oferecer um conjunto de ferramentas para compreender pontos fortes, fracos e oportunidades de melhoria.
\end{itemize}

Em essência, os MMs servem como um guia para a transformação, ajudando as empresas a identificar onde estão, para onde precisam ir e como chegar lá, especialmente em contextos de rápida mudança tecnológica e organizacional como a Indústria 4.0 e 5.0 [Gokalp2020].

\subsection{Objetivos da avaliação de maturidade em contextos industriais}

A avaliação de maturidade em contextos industriais, particularmente no cenário da Indústria 4.0 e 5.0, visa múltiplos objetivos estratégicos e operacionais. O principal propósito é capacitar as empresas a compreenderem sua posição atual em relação à adoção de tecnologias avançadas e princípios transformadores, como a digitalização, a automação e a colaboração humano-máquina [Lucato2019].

Entre os objetivos específicos, destacam-se:
\begin{itemize}
    \item \textbf{Diagnóstico e Posicionamento:} Identificar o nível atual de maturidade de uma organização em relação a um determinado paradigma industrial (e.g., Indústria 4.0 ou 5.0). Isso permite que as empresas compreendam suas lacunas e áreas de força [Schumacher2016].
    \item \textbf{Planejamento Estratégico:} Fornecer uma base sólida para o desenvolvimento de roteiros e planos de ação. Ao conhecer seu nível de maturidade, as organizações podem definir metas realistas e priorizar investimentos em tecnologias, processos e capacitação de pessoal [Mittal2018].
    \item \textbf{Benchmarking:} Comparar o desempenho da organização com o de concorrentes, líderes de mercado ou padrões da indústria. Isso ajuda a identificar as melhores práticas e a impulsionar a melhoria contínua [Angreani2020].
    \item \textbf{Gestão de Riscos:} Avaliar a prontidão da organização para enfrentar os desafios e riscos associados à transformação digital e à implementação de novas tecnologias, permitindo a mitigação proativa [Gokalp2020].
    \item \textbf{Otimização de Recursos:} Direcionar a alocação de recursos (financeiros, humanos e tecnológicos) de forma mais eficiente, garantindo que os investimentos sejam feitos nas áreas que trarão o maior impacto para a transição industrial [Ramos2020].
    \item \textbf{Fomento à Inovação e Cultura:} Estimular uma cultura de inovação e adaptabilidade, incentivando a força de trabalho a abraçar as mudanças e a desenvolver novas competências necessárias para o futuro da indústria [Berger2018].
\end{itemize}

Em suma, a avaliação de maturidade não é apenas um exercício de diagnóstico, mas uma ferramenta proativa para guiar as organizações através de transformações complexas, garantindo que elas permaneçam competitivas e relevantes em um cenário industrial em constante evolução [MouraKohl2020].

\subsection{Diferenças entre modelos genéricos e específicos}

Modelos de maturidade podem ser classificados em genéricos e específicos, cada um com suas próprias características, vantagens e desvantagens. A escolha entre um e outro depende do contexto, dos objetivos da avaliação e da profundidade de análise desejada [Angreani2020].

\textbf{Modelos Genéricos:}
Modelos genéricos são projetados para serem aplicáveis a uma ampla gama de organizações e setores, independentemente de sua área de atuação ou especificidades tecnológicas. Eles geralmente se concentram em princípios universais de gestão, processos e capacidades organizacionais. Exemplos notáveis incluem o Capability Maturity Model Integration (CMMI) para desenvolvimento de software e o Organizational Project Management Maturity Model (OPM3) para gestão de projetos [Angreani2020].

\begin{itemize}
    \item \textbf{Vantagens:} Ampla aplicabilidade, facilidade de comparação entre diferentes setores, e uma base conceitual bem estabelecida. Podem ser um bom ponto de partida para organizações que estão iniciando sua jornada de avaliação de maturidade.
    \item \textbf{Desvantagens:} Podem ser muito abstratos ou superficiais para capturar as nuances e os desafios específicos de um determinado setor ou tecnologia. A falta de especificidade pode levar a uma avaliação menos precisa e a recomendações menos acionáveis [Mittal2018].
\end{itemize}

\textbf{Modelos Específicos:}
Modelos específicos, por outro lado, são desenvolvidos para atender às necessidades e características de um setor, tecnologia ou domínio de aplicação particular. No contexto industrial, isso inclui modelos de maturidade focados na Indústria 4.0, como o modelo de Schumacher [Schumacher2016], Mittal [Mittal2018], e outros, que consideram dimensões como tecnologias habilitadoras, infraestrutura digital e integração de sistemas [Ramos2020].

\begin{itemize}
    \item \textbf{Vantagens:} Alta relevância e precisão para o domínio específico, fornecendo insights detalhados e recomendações altamente acionáveis. Eles incorporam o conhecimento especializado e as melhores práticas do setor, tornando a avaliação mais significativa [Gokalp2020].
    \item \textbf{Desvantagens:} Menor generalizabilidade, o que pode dificultar a comparação com organizações de outros setores. O desenvolvimento de modelos específicos pode ser mais complexo e exigir um profundo conhecimento do domínio [MouraKohl2020].
\end{itemize}

Para a avaliação da maturidade na Indústria 5.0, a tendência é a criação de modelos mais específicos que possam abordar as novas dimensões de human-centricity, sustentabilidade e resiliência, que não são totalmente contempladas pelos modelos genéricos ou mesmo pelos modelos da Indústria 4.0 [HeinPensel2023].

\subsection{Relação entre maturidade e transformação digital}

A relação entre maturidade e transformação digital é intrínseca e bidirecional. A transformação digital refere-se à integração de tecnologias digitais em todas as áreas de uma empresa, mudando fundamentalmente como ela opera e entrega valor aos clientes [Gokalp2020]. A maturidade, nesse contexto, é a capacidade de uma organização de planejar, executar e sustentar essa transformação de forma eficaz [Angreani2020].

\textbf{Maturidade como Habilitador da Transformação Digital:}
Uma organização com um alto nível de maturidade digital possui as capacidades necessárias para implementar e gerenciar iniciativas de transformação digital com sucesso. Isso inclui não apenas a adoção de tecnologias, mas também a adaptação de processos, a requalificação da força de trabalho e a mudança cultural. Empresas maduras digitalmente são mais ágeis, inovadoras e resilientes, o que as torna mais aptas a aproveitar as oportunidades da era digital [Berger2018].

\begin{itemize}
    \item \textbf{Visão e Estratégia:} Organizações maduras possuem uma visão clara e uma estratégia bem definida para a transformação digital, alinhando as iniciativas tecnológicas aos objetivos de negócio [Mittal2018].
    \item \textbf{Cultura e Liderança:} Uma cultura que abraça a experimentação, a colaboração e a aprendizagem contínua, juntamente com uma liderança engajada, são cruciais para impulsionar a transformação digital [MouraKohl2020].
    \item \textbf{Tecnologia e Dados:} A capacidade de integrar e alavancar tecnologias digitais, bem como de coletar, analisar e utilizar dados para a tomada de decisões, é um pilar fundamental da maturidade digital [Lucato2019].
    \item \textbf{Pessoas e Competências:} A requalificação e o desenvolvimento de novas competências na força de trabalho são essenciais para garantir que as pessoas estejam preparadas para operar em um ambiente digitalizado [Ramos2020].
\end{itemize}

\textbf{Transformação Digital como Impulsionador da Maturidade:}
Por outro lado, a própria jornada de transformação digital impulsiona a maturidade de uma organização. À medida que as empresas implementam novas tecnologias e adaptam seus processos, elas aprendem, evoluem e aprimoram suas capacidades. Cada iniciativa de transformação digital bem-sucedida contribui para elevar o nível de maturidade da organização, criando um ciclo virtuoso de melhoria contínua [Gajdzik2022].

No contexto da Indústria 5.0, essa relação se aprofunda. A Indústria 5.0, com seu foco em human-centricity, sustentabilidade e resiliência, exige uma transformação digital que vá além da eficiência e da produtividade, incorporando valores humanos e ambientais. A maturidade para a Indústria 5.0, portanto, envolve a capacidade de integrar essas novas dimensões na estratégia e nas operações digitais [BARO2025].





\section{Modelos de Maturidade na Indústria 4.0}

A Indústria 4.0, caracterizada pela fusão de tecnologias digitais e físicas, como Internet das Coisas (IoT), sistemas ciber-físicos (CPS) e inteligência artificial (IA), revolucionou a manufatura global [Lu2017]. Para auxiliar as empresas em sua jornada de transformação digital, diversos modelos de maturidade foram desenvolvidos, visando avaliar a prontidão e guiar a implementação das tecnologias e princípios da Indústria 4.0 [Angreani2020].

\subsection{Revisão de modelos consolidados para Indústria 4.0 }

Os modelos de maturidade para a Indústria 4.0 buscam fornecer uma estrutura para que as organizações avaliem seu estágio atual de adoção e identifiquem as próximas etapas para aprimorar suas capacidades. Vários modelos se consolidaram na literatura, cada um com suas particularidades e focos.

\subsubsection{Modelo de Schumacher}

O modelo de maturidade proposto por Schumacher, Erol e Sihn (2016) [Schumacher2016] é um dos mais referenciados na avaliação da prontidão para a Indústria 4.0. Ele define seis dimensões principais para a avaliação da maturidade de empresas manufatureiras:
\begin{itemize}
    \item \textbf{Estratégia e Organização:} Avalia a existência de uma estratégia clara para a Indústria 4.0 e a estrutura organizacional para suportá-la.
    \item \textbf{Fábrica Inteligente:} Foca na digitalização e interconexão dos processos de produção, incluindo o uso de CPS e IoT.
    \item \textbf{Operações Inteligentes:} Analisa a otimização dos processos operacionais através da análise de dados e automação.
    \item \textbf{Produtos Inteligentes:} Refere-se à capacidade de desenvolver produtos conectados e serviços digitais.
    \item \textbf{Pessoas e Cultura:} Avalia a prontidão da força de trabalho, as competências digitais e a cultura organizacional para a mudança.
    \item \textbf{Tecnologia:} Examina a infraestrutura tecnológica existente e a capacidade de integrar novas tecnologias.
\end{itemize}

O modelo de Schumacher estabelece cinco níveis de maturidade, desde o iniciante até o líder, fornecendo um roteiro claro para a evolução das empresas. Sua abordagem holística e as dimensões bem definidas o tornaram uma referência importante para o diagnóstico da Indústria 4.0.

\subsubsection{Modelo de Mittal}

Mittal et al. (2018) [Mittal2018] realizaram uma revisão crítica dos modelos de maturidade para a Indústria 4.0, com foco nas implicações para pequenas e médias empresas (PMEs). Embora não proponham um modelo único, eles analisam as dimensões e características comuns em diversos modelos existentes, destacando a importância de uma abordagem adaptada às necessidades das PMEs.

Eles enfatizam que muitos modelos são complexos e caros para PMEs, sugerindo a necessidade de modelos mais simplificados e práticos. As dimensões frequentemente abordadas nos modelos analisados por Mittal incluem tecnologia, organização, pessoas, processos e estratégia, ressaltando a interconexão entre esses pilares para uma transformação bem-sucedida. A revisão de Mittal é crucial para entender as lacunas e desafios na aplicação de modelos de maturidade em diferentes contextos empresariais.

\subsubsection{Outros Modelos Relevantes (Ramos, Gokalp, Berger, Moura Kohl)}

Além dos modelos de Schumacher e das análises de Mittal, outros pesquisadores contribuíram significativamente para o campo dos modelos de maturidade da Indústria 4.0:
\begin{itemize}
    \item \textbf{Ramos e Lima (2020) [Ramos2020]:} Realizaram uma análise de modelos de maturidade e avaliação do estado atual das organizações para a implementação da Indústria 4.0. Eles destacam a importância de modelos que considerem as especificidades de cada organização e setor, e que forneçam um roteiro claro para a implementação.
    \item \textbf{Gokalp et al. (2020) [Gokalp2020]:} Propuseram um modelo de maturidade de capacidade de transformação digital que permite a avaliação de fabricantes industriais. O modelo enfatiza a importância da digitalização em todas as áreas da empresa, incluindo processos, pessoas e tecnologia, para alcançar a maturidade digital.
    \item \textbf{Berger et al. (2018) [Berger2018]:} Focaram na contextualização do resultado de uma avaliação de maturidade para a Indústria 4.0, ressaltando que a interpretação dos resultados deve levar em conta o contexto específico da organização. Eles abordam dimensões como estratégia, organização, tecnologia e cultura.
    \item \textbf{Moura e Kohl (2020) [MouraKohl2020]:} Apresentaram uma avaliação comparativa de modelos de maturidade na Indústria 4.0, analisando suas semelhanças e diferenças em termos de dimensões, níveis de maturidade e aplicabilidade. Eles reforçam a ideia de que não existe um modelo 'tamanho único', e a escolha deve ser baseada nas necessidades da empresa.
\end{itemize}

Esses modelos, em conjunto, demonstram a diversidade de abordagens e a complexidade envolvida na avaliação da maturidade para a Indústria 4.0, cada um contribuindo com perspectivas valiosas para a compreensão e implementação dessa transformação.

\subsection{Dimensões típicas utilizadas (tecnologia, organização, pessoas, processos, etc.)}

Os modelos de maturidade para a Indústria 4.0, embora variem em sua estrutura e foco, geralmente convergem em um conjunto de dimensões-chave que são consideradas essenciais para uma transformação digital bem-sucedida. Angreani et al. (2020) [Angreani2020] identificaram nove categorias de dimensões frequentemente utilizadas:
\begin{itemize}
    \item \textbf{Estratégia:} Refere-se à existência de uma visão clara e objetivos definidos para a implementação da Indústria 4.0, alinhados aos objetivos de negócio da organização.
    \item \textbf{Liderança:} Avalia o engajamento e o apoio da alta gerência na condução da transformação digital, bem como a capacidade de inspirar e guiar a equipe.
    \item \textbf{Clientes:} Foca na forma como a Indústria 4.0 impacta a experiência do cliente, a personalização de produtos e serviços, e a interação com o mercado.
    \item \textbf{Produtos:} Diz respeito à capacidade de desenvolver e oferecer produtos e serviços inteligentes, conectados e baseados em dados.
    \item \textbf{Operações:} Abrange a digitalização e otimização dos processos de produção, logística e cadeia de suprimentos, incluindo automação e análise de dados em tempo real.
    \item \textbf{Cultura:} Avalia a adaptabilidade da cultura organizacional, a abertura à inovação, a colaboração e a disposição para a mudança.
    \item \textbf{Pessoas:} Foca nas competências da força de trabalho, na necessidade de requalificação e no desenvolvimento de novas habilidades para operar em um ambiente digitalizado.
    \item \textbf{Governança:} Refere-se às estruturas de gestão, políticas e regulamentações internas que suportam a implementação da Indústria 4.0, incluindo segurança de dados e conformidade.
    \item \textbf{Tecnologia:} Examina a infraestrutura tecnológica existente, a adoção de tecnologias habilitadoras (IoT, IA, Big Data, Cloud Computing) e a capacidade de integração de sistemas.
\end{itemize}

As dimensões de tecnologia e operações são as mais frequentemente abordadas na maioria dos modelos, refletindo o foco inicial da Indústria 4.0 na automação e digitalização da produção. No entanto, a importância de dimensões como estratégia, pessoas e cultura tem sido cada vez mais reconhecida, pois são cruciais para o sucesso a longo prazo da transformação digital [Angreani2020].

\subsection{Limitações dos Modelos para o Contexto da Indústria 5.0}

Embora os modelos de maturidade da Indústria 4.0 tenham sido fundamentais para guiar a transformação digital das empresas, eles apresentam limitações significativas quando aplicados ao contexto emergente da Indústria 5.0. A Indústria 5.0, com seus pilares de human-centricity, sustentabilidade e resiliência, transcende o foco predominantemente tecnológico e de eficiência da Indústria 4.0 [BARO2025].

As principais limitações dos modelos existentes incluem:
\begin{itemize}
    \item \textbf{Foco Tecnológico Excessivo:} Muitos modelos da Indústria 4.0 são fortemente orientados para a tecnologia e a automação, negligenciando as dimensões sociais, humanas e ambientais que são centrais para a Indústria 5.0 [HeinPensel2023]. Eles tendem a ver a tecnologia como um fim em si mesma, e não como um meio para alcançar objetivos mais amplos relacionados ao bem-estar humano e à sustentabilidade.
    \item \textbf{Lacuna em Human-Centricity:} Os modelos da Indústria 4.0 frequentemente não abordam adequadamente a colaboração humano-máquina, o desenvolvimento de habilidades humanas em conjunto com a automação, a ergonomia e o bem-estar dos trabalhadores. A Indústria 5.0, por outro lado, coloca o ser humano no centro do processo produtivo, exigindo que os modelos de maturidade avaliem a integração harmoniosa entre humanos e sistemas inteligentes [BARO2025].
    \item \textbf{Insuficiência em Sustentabilidade:} Embora alguns modelos da Indústria 4.0 toquem em aspectos de eficiência de recursos, eles raramente incorporam uma visão holística da sustentabilidade, incluindo economia circular, uso de energias renováveis, materiais eco-friendly e a pegada ambiental completa. A Indústria 5.0 exige que a sustentabilidade seja um pilar fundamental, e não um mero subproduto da eficiência [HeinPensel2023].
    \item \textbf{Deficiência em Resiliência:} A resiliência, a capacidade de um sistema de se adaptar e se recuperar de interrupções, é um aspecto crítico da Indústria 5.0, especialmente em face de desafios globais como pandemias e crises climáticas. Os modelos da Indústria 4.0 tendem a focar mais na otimização da cadeia de suprimentos e na eficiência, sem aprofundar-se na robustez e adaptabilidade necessárias para a Indústria 5.0 [BARO2025].
    \item \textbf{Falta de Critérios Éticos e Sociais:} A Indústria 5.0 levanta questões éticas complexas relacionadas à IA, privacidade de dados e impacto social da automação. Os modelos da Indústria 4.0 geralmente não possuem critérios robustos para avaliar a governança ética e o impacto social das tecnologias [HeinPensel2023].
\end{itemize}

Em suma, a transição para a Indústria 5.0 exige uma reavaliação e expansão dos modelos de maturidade existentes, para que possam abranger as novas dimensões e princípios que definem essa próxima fase da evolução industrial [HeinPensel2023].





\section{Avanços e Desafios para a Indústria 5.0}

Enquanto a Indústria 4.0 se concentrou na digitalização e automação para otimizar a produção e a eficiência, uma nova fase, a Indústria 5.0, emerge com um foco expandido que transcende a mera tecnologia. Esta nova era busca integrar o elemento humano, a sustentabilidade e a resiliência como pilares centrais, respondendo a desafios sociais e ambientais que a Indústria 4.0 não abordou completamente [BARO2025].

\subsection{Conceito e princípios da Indústria 5.0 (human-centric, sustainable, resilient)}

A Indústria 5.0 é uma visão que complementa e expande a Indústria 4.0, colocando o bem-estar humano e a sustentabilidade no centro da produção industrial. Ela é impulsionada pela necessidade de uma indústria mais centrada no ser humano, ecologicamente responsável e capaz de resistir a choques externos. Os três pilares fundamentais da Indústria 5.0 são [Breque2021]:

\begin{itemize}
    \item \textbf{Human-Centric (Centrada no Ser Humano):} Este princípio enfatiza a colaboração harmoniosa entre humanos e máquinas inteligentes. Em vez de substituir o trabalho humano, a Indústria 5.0 busca capacitar os trabalhadores, aprimorando suas habilidades e criando ambientes de trabalho mais seguros, ergonômicos e satisfatórios. O foco é na personalização da experiência de trabalho, no desenvolvimento de novas competências e na garantia de que a tecnologia sirva ao bem-estar humano [BARO2025].
    \item \textbf{Sustainable (Sustentável):} A Indústria 5.0 promove a produção industrial que respeita os limites planetários. Isso envolve a otimização do uso de recursos, a transição para fontes de energia renováveis, a implementação de princípios de economia circular (redução, reutilização, reciclagem) e a minimização do impacto ambiental ao longo de todo o ciclo de vida do produto. A sustentabilidade não é vista como um custo adicional, mas como um valor intrínseco e um impulsionador de inovação [HeinPensel2023].
    \item \textbf{Resilient (Resiliente):} Este pilar refere-se à capacidade das indústrias de se adaptarem e se recuperarem rapidamente de interrupções inesperadas, como desastres naturais, pandemias ou crises econômicas. A resiliência é alcançada através da flexibilidade dos sistemas de produção, da diversificação das cadeias de suprimentos, da implementação de tecnologias adaptativas e da tomada de decisões baseada em dados em tempo real para garantir a continuidade das operações [BARO2025].
\end{itemize}

Esses princípios visam criar um ecossistema industrial que não apenas seja eficiente e produtivo, mas também socialmente responsável e ambientalmente consciente, contribuindo para uma sociedade mais justa e um futuro mais sustentável [Zizic2022].

\subsection{Lacunas nos modelos existentes diante dos novos paradigmas}

Os modelos de maturidade desenvolvidos para a Indústria 4.0, embora eficazes em seu propósito original, revelam lacunas significativas quando confrontados com os novos paradigmas da Indústria 5.0. A principal razão para essas lacunas reside na diferença fundamental de foco entre as duas eras industriais [HeinPensel2023].

As principais lacunas incluem:
\begin{itemize}
    \item \textbf{Foco Limitado no Ser Humano:} Modelos da Indústria 4.0 tendem a abordar o fator humano principalmente em termos de competências digitais e adaptação à automação. Eles não se aprofundam em aspectos como o bem-estar do trabalhador, a ergonomia avançada, a colaboração humano-máquina em um nível mais intrínseco, a personalização do ambiente de trabalho e a ética na interação com sistemas inteligentes. A Indústria 5.0 exige uma avaliação mais profunda da 


integração homem-máquina e do impacto da tecnologia na qualidade de vida do trabalhador [BARO2025].
    \item \textbf{Abordagem Insuficiente da Sustentabilidade:} Embora a Indústria 4.0 tenha trazido ganhos de eficiência que podem indiretamente contribuir para a sustentabilidade, seus modelos de maturidade não a estabelecem como um pilar central. Aspectos como a economia circular, a gestão de resíduos, o uso de energias renováveis e a avaliação do ciclo de vida completo dos produtos são frequentemente secundários ou ausentes. A Indústria 5.0, ao contrário, exige que a sustentabilidade seja uma dimensão avaliada de forma proativa e abrangente [HeinPensel2023].
    \item \textbf{Resiliência Subestimada:} A Indústria 4.0 foca na otimização e na eficiência da cadeia de suprimentos, mas a resiliência em face de disrupções sistêmicas (como pandemias, crises geopolíticas ou desastres naturais) não é uma dimensão central de seus modelos de maturidade. A Indústria 5.0, por sua vez, exige que as organizações sejam capazes de se adaptar e se recuperar rapidamente, o que implica em avaliações de flexibilidade da produção, redundância de sistemas e capacidade de resposta a crises [BARO2025].
    \item \textbf{Falta de Critérios Éticos e de Governança Social:} Com o avanço da inteligência artificial e da automação, surgem questões éticas complexas, como viés algorítmico, privacidade de dados e o impacto social da automação no emprego. Os modelos da Indústria 4.0 não incorporam adequadamente esses critérios éticos e de governança social, que são fundamentais para a Indústria 5.0 [HeinPensel2023].
    \item \textbf{Foco na Eficiência vs. Valor Agregado Humano:} A Indústria 4.0 prioriza a eficiência e a produtividade. A Indústria 5.0, embora valorize a eficiência, busca um valor agregado que inclua o bem-estar do trabalhador e o impacto positivo na sociedade. Os modelos existentes não medem adequadamente esse valor agregado humano e social [Zizic2022].
\end{itemize}

Essas lacunas demonstram a necessidade de novos modelos de maturidade ou de uma adaptação significativa dos existentes para que possam capturar a complexidade e as prioridades da Indústria 5.0.

\subsection{Exigência de novas dimensões e critérios para avaliação}

A transição para a Indústria 5.0 impõe a necessidade de incorporar novas dimensões e critérios nos modelos de avaliação de maturidade, que vão além do escopo predominantemente tecnológico da Indústria 4.0. Para que uma avaliação seja verdadeiramente relevante e eficaz no contexto da Indústria 5.0, ela deve abranger os pilares de human-centricity, sustentabilidade e resiliência de forma explícita e mensurável [BARO2025].

As novas dimensões e critérios exigidos incluem:
\begin{itemize}
    \item \textbf{Dimensões de Human-Centricity:}
    \begin{itemize}
        \item \textit{Colaboração Humano-Máquina Avançada:} Avaliação da integração de sistemas inteligentes que aprimoram as capacidades humanas, em vez de apenas substituí-las. Isso inclui interfaces intuitivas, sistemas de apoio à decisão e robótica colaborativa (cobots) [BARO2025].
        \item \textit{Desenvolvimento de Habilidades e Requalificação (Upskilling/Reskilling):} Medição da capacidade da organização em identificar, desenvolver e adaptar as competências da força de trabalho para operar em ambientes híbridos humano-máquina, com foco em habilidades cognitivas, criatividade e resolução de problemas [HeinPensel2023].
        \item \textit{Bem-Estar e Ergonomia no Trabalho:} Critérios que avaliam o impacto das tecnologias no bem-estar físico e mental dos trabalhadores, incluindo ergonomia, redução de tarefas repetitivas e perigosas, e promoção de um ambiente de trabalho saudável e seguro [BARO2025].
        \item \textit{Personalização da Experiência de Trabalho:} Avaliação de como a tecnologia é utilizada para adaptar o trabalho às preferências e necessidades individuais dos funcionários, promovendo flexibilidade e satisfação [BARO2025].
        \item \textit{Ética e Governança da IA:} Critérios para avaliar a existência de políticas e práticas que garantam o uso ético da inteligência artificial, a proteção da privacidade dos dados dos trabalhadores e a mitigação de vieses algorítmicos [HeinPensel2023].
    \end{itemize}
    \item \textbf{Dimensões de Sustentabilidade:}
    \begin{itemize}
        \item \textit{Economia Circular e Gestão de Recursos:} Avaliação da implementação de estratégias de economia circular, como design para durabilidade, reutilização, reparo e reciclagem de produtos e materiais, e otimização do uso de recursos (água, energia, matérias-primas) [BARO2025].
        \item \textit{Energia Renovável e Eficiência Energética:} Medição da adoção de fontes de energia limpa e da implementação de tecnologias para otimizar o consumo de energia em toda a cadeia de valor [HeinPensel2023].
        \item \textit{Pegada Ambiental e Avaliação do Ciclo de Vida (ACV):} Critérios para quantificar e reduzir o impacto ambiental dos produtos e processos ao longo de todo o seu ciclo de vida, desde a extração da matéria-prima até o descarte [BARO2025].
        \item \textit{Cadeia de Suprimentos Verde:} Avaliação da sustentabilidade das práticas de sourcing, logística e colaboração com fornecedores e parceiros para minimizar o impacto ambiental [HeinPensel2023].
    \end{itemize}
    \item \textbf{Dimensões de Resiliência:}
    \begin{itemize}
        \item \textit{Flexibilidade e Adaptabilidade da Produção:} Medição da capacidade dos sistemas de produção de se reconfigurarem rapidamente em resposta a mudanças na demanda, interrupções na cadeia de suprimentos ou eventos inesperados [BARO2025].
        \item \textit{Gestão de Riscos e Continuidade de Negócios:} Avaliação da robustez dos planos de contingência, da capacidade de resposta a crises e da implementação de tecnologias que garantam a continuidade das operações em cenários adversos [HeinPensel2023].
        \item \textit{Cibersegurança e Proteção de Dados:} Critérios aprimorados para avaliar a segurança de sistemas interconectados e a proteção de dados sensíveis contra ataques cibernéticos e falhas [BARO2025].
        \item \textit{Resiliência da Cadeia de Suprimentos:} Avaliação da capacidade da cadeia de suprimentos de absorver choques, se recuperar e se adaptar a disrupções, incluindo a diversificação de fornecedores e o uso de tecnologias de rastreamento [HeinPensel2023].
        \item \textit{Tomada de Decisão Baseada em Dados em Tempo Real:} Medição da capacidade de coletar, analisar e agir sobre dados em tempo real para tomar decisões rápidas e eficazes durante períodos de incerteza ou disrupção [BARO2025].
    \end{itemize}
\end{itemize}

A inclusão dessas novas dimensões é crucial para que os modelos de maturidade possam efetivamente guiar as organizações na complexa transição para a Indústria 5.0, garantindo que a tecnologia sirva a propósitos mais amplos de bem-estar social e ambiental.





\section{Propostas de Modelos para a Indústria 5.0}

Com o surgimento da Indústria 5.0 e seus novos pilares de human-centricity, sustentabilidade e resiliência, a necessidade de modelos de maturidade que capturem essas dimensões tornou-se premente. Embora a pesquisa ainda esteja em estágios iniciais, algumas propostas de modelos de avaliação para a Indústria 5.0 já começaram a surgir na literatura, buscando preencher as lacunas deixadas pelos modelos da Indústria 4.0 [HeinPensel2023].

\subsection{Análise das propostas recentes}

As propostas recentes de modelos de maturidade para a Indústria 5.0 buscam ir além do foco tecnológico da Indústria 4.0, incorporando as novas dimensões que definem essa próxima fase da evolução industrial. É importante notar que muitos desses modelos ainda estão em fase conceitual ou inicial de validação.

\subsubsection{Modelo de Pizon et al. (2023)}

Embora o documento fornecido não contenha detalhes específicos sobre um modelo de maturidade proposto por Pizon et al. (2023), a literatura geral sobre Indústria 5.0 sugere que novas abordagens estão sendo desenvolvidas para avaliar a integração de fatores humanos, ambientais e de resiliência. É provável que um modelo de Pizon et al. se concentre em como as empresas podem medir e aprimorar sua capacidade de alinhar a tecnologia com o bem-estar dos trabalhadores e a sustentabilidade, bem como sua capacidade de se adaptar a choques externos. Tais modelos frequentemente incorporam métricas qualitativas e quantitativas para avaliar a prontidão organizacional em relação a esses novos paradigmas.

\subsubsection{Modelo de Rame (2024)}

Similarmente ao caso de Pizon et al., o material de referência não detalha um modelo específico de Rame (2024). No entanto, a tendência geral na pesquisa da Indústria 5.0 aponta para modelos que enfatizam a colaboração humano-máquina, a ética na inteligência artificial, a gestão de recursos e a resiliência da cadeia de suprimentos. Um modelo de Rame provavelmente abordaria a capacidade de uma organização de integrar esses elementos de forma coesa, avaliando não apenas a adoção de tecnologias, mas também a transformação cultural e estratégica necessária para a Indústria 5.0. O foco estaria em como as empresas podem criar um ambiente de trabalho mais humano e sustentável, ao mesmo tempo em que garantem a robustez de suas operações.

\subsubsection{Modelos de Yang (2024) e Toth (2023)}

Os documentos fornecidos não contêm informações detalhadas sobre modelos de maturidade específicos de Yang (2024) e Toth (2023) para a Indústria 5.0. No entanto, a pesquisa contemporânea sobre a Indústria 5.0, como a de Baro et al. (2025) [BARO2025] e Hein-Pensel et al. (2023) [HeinPensel2023], indica que os modelos emergentes estão focando em:
\begin{itemize}
    \item \textbf{Integração Socio-Técnica:} Avaliando a harmonização entre os sistemas sociais (pessoas, cultura, organização) e os sistemas técnicos (tecnologia, processos) para otimizar o desempenho geral e o bem-estar.
    \item \textbf{Dimensões Expandidas:} Incluindo critérios para human-centricity (colaboração humano-máquina, desenvolvimento de habilidades, bem-estar), sustentabilidade (economia circular, energia renovável, pegada ambiental) e resiliência (flexibilidade, gestão de riscos, cibersegurança).
    \item \textbf{Abordagem Holística:} Considerar a Indústria 5.0 como um sistema complexo onde todos os elementos estão interconectados, exigindo uma avaliação abrangente que vá além da eficiência de produção.
\end{itemize}

É razoável inferir que os modelos de Yang e Toth, se existirem e forem relevantes para o contexto, seguiriam essas tendências, contribuindo para a compreensão de como as organizações podem medir seu progresso em direção a uma Indústria 5.0 mais madura.

\subsection{Limitações identificadas quanto à aplicabilidade e objetividade}

Embora as propostas de modelos para a Indústria 5.0 sejam um passo crucial para guiar as organizações nessa nova era, elas ainda enfrentam diversas limitações em termos de aplicabilidade e objetividade. Essas limitações são inerentes ao estágio inicial de desenvolvimento do conceito da Indústria 5.0 e à complexidade das dimensões que ela abrange [HeinPensel2023].

As principais limitações incluem:
\begin{itemize}
    \item \textbf{Falta de Validação Empírica Robusta:} Muitos dos modelos propostos são de natureza conceitual ou foram validados em estudos de caso limitados. Há uma carência de validação empírica em larga escala que comprove sua eficácia e aplicabilidade em diversos setores e tamanhos de empresas. Isso dificulta a generalização dos resultados e a confiança na sua capacidade de prever o sucesso da implementação da Indústria 5.0 [HeinPensel2023].
    \item \textbf{Subjetividade na Avaliação de Novas Dimensões:} As dimensões de human-centricity, sustentabilidade e resiliência, embora cruciais, são inerentemente mais subjetivas e difíceis de quantificar do que as métricas de eficiência e tecnologia da Indústria 4.0. A avaliação de aspectos como 'bem-estar do trabalhador' ou 'cultura de resiliência' pode variar significativamente entre avaliadores, comprometendo a objetividade dos resultados [BARO2025].
    \item \textbf{Complexidade e Abrangência Excessiva:} A tentativa de incorporar todas as novas dimensões da Indústria 5.0 pode resultar em modelos excessivamente complexos e difíceis de implementar. A coleta de dados e a análise de tantos critérios podem ser onerosas e consumir muitos recursos, especialmente para PMEs [HeinPensel2023].
    \item \textbf{Generalizabilidade Limitada:} Assim como nos modelos da Indústria 4.0, a aplicabilidade dos modelos da Indústria 5.0 pode ser limitada a setores ou tipos específicos de organizações. Um modelo que funciona bem para uma grande corporação de manufatura pode não ser adequado para uma PME de serviços, por exemplo. A falta de adaptabilidade a diferentes contextos organizacionais é uma barreira à sua ampla adoção [BARO2025].
    \item \textbf{Dinâmica da Indústria 5.0:} A Indústria 5.0 é um conceito em evolução, com novas tecnologias e práticas surgindo constantemente. Isso significa que os modelos de maturidade podem se tornar rapidamente desatualizados, exigindo revisões e atualizações contínuas, o que representa um desafio para sua manutenção e relevância a longo prazo [HeinPensel2023].
    \item \textbf{Ausência de Métricas Padronizadas:} Ainda não há um consenso sobre as métricas e indicadores padronizados para avaliar a maturidade em relação aos pilares da Indústria 5.0. A falta de padronização dificulta a comparação entre diferentes avaliações e a criação de benchmarks confiáveis [BARO2025].
\end{itemize}

Superar essas limitações exigirá um esforço colaborativo entre pesquisadores e a indústria para desenvolver modelos mais práticos, objetivos e validados empiricamente, que possam efetivamente guiar a transição para a Indústria 5.0.





\section{Critérios para Avaliação de Maturidade na Indústria 5.0}

A avaliação da maturidade na Indústria 5.0 exige uma redefinição dos critérios tradicionais, que eram predominantemente focados em tecnologia e eficiência. Com a introdução dos pilares de human-centricity, sustentabilidade e resiliência, os novos modelos de maturidade devem incorporar uma gama mais ampla de indicadores que reflitam esses valores. A identificação e classificação desses critérios são fundamentais para desenvolver ferramentas de avaliação eficazes e abrangentes [BARO2025].

\subsection{Identificação dos Critérios Relevantes}

Os critérios relevantes para a avaliação de maturidade na Indústria 5.0 derivam diretamente de seus três pilares fundamentais, expandindo o escopo das avaliações da Indústria 4.0. Eles devem ser capazes de medir não apenas a adoção tecnológica, mas também o impacto social, ambiental e a capacidade de adaptação da organização [HeinPensel2023].

\textbf{Critérios Relacionados à Human-Centricity:}
\begin{itemize}
    \item \textbf{Colaboração Humano-Máquina:} Avalia a qualidade e a eficácia da interação entre trabalhadores e sistemas autônomos/inteligentes. Critérios incluem a facilidade de uso das interfaces, a capacidade de sistemas de IA de apoiar a tomada de decisão humana, e a presença de robôs colaborativos (cobots) que aumentam as capacidades humanas [BARO2025].
    \item \textbf{Desenvolvimento de Competências:} Mede a proatividade da organização em requalificar e aprimorar as habilidades da força de trabalho para operar em ambientes complexos e dinâmicos. Inclui programas de treinamento em novas tecnologias, habilidades socioemocionais e pensamento crítico [HeinPensel2023].
    \item \textbf{Bem-Estar e Ergonomia:} Foca na criação de ambientes de trabalho seguros, saudáveis e ergonômicos. Critérios podem incluir a redução de acidentes, a minimização de estresse relacionado ao trabalho, e a implementação de soluções tecnológicas que melhoram o conforto e a segurança dos trabalhadores [BARO2025].
    \item \textbf{Personalização do Trabalho:} Avalia a flexibilidade e a adaptabilidade das práticas de trabalho para atender às necessidades individuais dos funcionários, promovendo um melhor equilíbrio entre vida profissional e pessoal [BARO2025].
    \item \textbf{Ética e Governança:} Examina a existência de políticas claras e mecanismos de controle para garantir o uso ético da IA, a proteção da privacidade dos dados dos funcionários e a prevenção de vieses algorítmicos [HeinPensel2023].
\end{itemize}

\textbf{Critérios Relacionados à Sustentabilidade:}
\begin{itemize}
    \item \textbf{Economia Circular:} Mede a implementação de práticas que promovem a redução, reutilização e reciclagem de materiais e produtos. Critérios incluem o design de produtos para longevidade, a minimização de resíduos e a recuperação de recursos [BARO2025].
    \item \textbf{Eficiência Energética e Fontes Renováveis:} Avalia o consumo de energia e a proporção de energia proveniente de fontes renováveis. Inclui a otimização de processos para reduzir o consumo e o investimento em infraestrutura de energia limpa [HeinPensel2023].
    \item \textbf{Pegada Ambiental:} Quantifica o impacto ambiental geral das operações, incluindo emissões de carbono, consumo de água e geração de resíduos. Critérios podem envolver a realização de Avaliações de Ciclo de Vida (ACV) para produtos [BARO2025].
    \item \textbf{Cadeia de Suprimentos Sustentável:} Foca na adoção de práticas sustentáveis em toda a cadeia de valor, desde a seleção de fornecedores até a logística de distribuição, garantindo a responsabilidade social e ambiental [HeinPensel2023].
\end{itemize}

\textbf{Critérios Relacionados à Resiliência:}
\begin{itemize}
    \item \textbf{Flexibilidade e Adaptabilidade Operacional:} Mede a capacidade da organização de ajustar rapidamente seus processos de produção e cadeia de suprimentos em resposta a disrupções. Critérios incluem a modularidade dos sistemas e a agilidade na tomada de decisões [BARO2025].
    \item \textbf{Gestão de Riscos e Continuidade de Negócios:} Avalia a robustez dos planos de contingência, a capacidade de identificar e mitigar riscos, e a implementação de tecnologias que garantam a continuidade das operações em cenários adversos [HeinPensel2023].
    \item \textbf{Cibersegurança:} Foca na proteção de sistemas e dados contra ameaças cibernéticas, garantindo a integridade e a disponibilidade das informações em um ambiente altamente conectado [BARO2025].
    \item \textbf{Tomada de Decisão Baseada em Dados:} Mede a capacidade de coletar, analisar e utilizar dados em tempo real para informar decisões estratégicas e operacionais, especialmente durante períodos de incerteza [HeinPensel2023].
\end{itemize}

Esses critérios, quando avaliados de forma integrada, fornecem uma visão abrangente da maturidade de uma organização em relação aos princípios da Indústria 5.0.

\subsection{Classificação e Natureza dos Critérios(qualitativos, quantitativos, subjetivos)}

Os critérios de avaliação de maturidade na Indústria 5.0 podem ser classificados de acordo com sua natureza, abrangendo indicadores qualitativos, quantitativos e, em alguns casos, subjetivos. A combinação desses diferentes tipos de critérios é essencial para capturar a complexidade e a multidimensionalidade da Indústria 5.0 [BARO2025].

\textbf{Critérios Quantitativos:}
São aqueles que podem ser medidos numericamente e expressos em valores objetivos. Eles são frequentemente utilizados para avaliar aspectos relacionados à eficiência, desempenho e impacto mensurável. Exemplos incluem:
\begin{itemize}
    \item \textit{Eficiência Energética:} Consumo de energia por unidade de produção (kWh/peça).
    \item \textit{Redução de Resíduos:} Percentual de resíduos reciclados ou reutilizados.
    \item \textit{Taxa de Acidentes de Trabalho:} Número de acidentes por milhão de horas trabalhadas.
    \item \textit{Investimento em P&D:} Percentual do faturamento investido em pesquisa e desenvolvimento de tecnologias da Indústria 5.0.
    \item \textit{Tempo de Inatividade do Sistema:} Horas de parada não planejada de equipamentos ou sistemas.
\end{itemize}

Esses critérios oferecem uma base sólida para o benchmarking e permitem o acompanhamento do progresso ao longo do tempo de forma clara e verificável [HeinPensel2023].

\textbf{Critérios Qualitativos:}
São descritivos e baseados em observações, entrevistas e análises de documentos. Eles são cruciais para avaliar aspectos que não são facilmente quantificáveis, como cultura organizacional, processos de tomada de decisão e a qualidade das interações. Exemplos incluem:
\begin{itemize}
    \item \textit{Cultura de Inovação:} Avaliação da abertura da organização a novas ideias e experimentação.
    \item \textit{Engajamento dos Funcionários:} Percepção dos trabalhadores sobre seu envolvimento e valorização na empresa.
    \item \textit{Qualidade da Colaboração Humano-Máquina:} Análise da fluidez e eficácia da interação entre humanos e sistemas inteligentes.
    \item \textit{Mecanismos de Governança Ética:} Existência e eficácia de comitês ou políticas para lidar com questões éticas da IA.
    \item \textit{Flexibilidade Organizacional:} Capacidade da estrutura organizacional de se adaptar a mudanças rápidas no ambiente de negócios [BARO2025].
\end{itemize}

Critérios qualitativos fornecem profundidade e contexto aos dados quantitativos, permitindo uma compreensão mais rica da maturidade da organização [HeinPensel2023].

\textbf{Critérios Subjetivos:}
Embora muitas vezes se sobreponham aos qualitativos, os critérios subjetivos dependem fortemente da percepção e do julgamento individual dos avaliadores ou dos membros da organização. Eles são particularmente relevantes para avaliar aspectos relacionados a sentimentos, percepções e valores. Exemplos incluem:
\begin{itemize}
    \item \textit{Satisfação do Trabalhador:} Avaliada através de pesquisas de clima organizacional e feedback direto.
    \item \textit{Percepção de Resiliência:} Como os líderes e funcionários percebem a capacidade da organização de lidar com crises.
    \item \textit{Nível de Confiança:} A confiança entre os diferentes níveis hierárquicos e entre humanos e sistemas autônomos.
\end{itemize}

É importante que a avaliação de critérios subjetivos seja complementada por metodologias robustas, como escalas de Likert, entrevistas estruturadas e grupos focais, para minimizar vieses e garantir a consistência [BARO2025].

A combinação equilibrada desses três tipos de critérios permite uma avaliação holística e precisa da maturidade na Indústria 5.0, fornecendo insights acionáveis para a jornada de transformação das organizações.
